% !TEX root =  ../geospatial-video.tex

\section{User Interface}
Although we did not implement a user interface for \todo{TOOL NAME}, we were able to prototype some designs for such an interface. Based on these explorations, we propose a block-based projectional editor interface to provide the most convenience and accessibility for programmers, particularly those with less experience who we expect to use this tool. To facilitate this, we would allow users to fill the holes in skeleton queries by selecting entries from a drop-down menu which would also have a search bar; this will provide suggestions to users who want to explore the dataset, but also allow users who know what want to see to search for what they are looking for.  

As the output of a query in this language would be a series of clips or videos taken from the overall collection of videos, we envision the interface as having some region in which the output of the queries are displayed. However, this creates some issues, as video data is large. To amend this, we would \todo{how will we display videos? ask Shadaj?}

\subsection{Mapping the Language to the Interface}


\todo{to focus: \\
- how we map the programming language to the user interface. For example, how do we present our data model visually in the graphical user tool. \\
- how do the queries that we can construct in the graphical user tool reflect the queries that we can construct with our query language. \\
- We do not have to try to cover all the cases that the query language can produce, but we should try to capture the one that are important.
For example, how to query by frames, how to query by instances, how to represent the lambda function, how to represent the custom function.
} \\
\todo{Andrew} \\
\todo{Mick}