% !TEX root =  ../geospatial-video.tex

\section{Introduction}
\mick{todo - make sure to address motivation as well \\
- I (Andrew) gave a rough answer to some of the questions posed in the guidelines, but they could probably use another look\\
- Why is this area important/why should a reader care about the work you did?\\
We want to make it easier to search through massive video datasets, especially given how important the research that users do on these datasets is (TODO: Fix this sentence!!!) Data journalists often have large banks of video data, but not nearly enough time to look through this data, which can make it hard to find evidence of police misconduct. Machine learning researchers who work with autonomous vehicles often want to find the most tricky scenarios in order to train the vehicles to maneuver such situations. [further explanation needed] \\
- What will be the outcome if your project is successful?\\
We hope to create a tool that will allow programmers of any experience level to easily explore large video databases.
- Why hasn’t this topic been studied before, or why haven’t prior studies answered the questions you’re answering? \\
One issue with developing this tool is that in order to use it, there must be an annotated dataset that contains information such as the geographical location of objects and labels of the elements in the video. However, some of these concerns could be amended by using a computer vision model to label the video, so we will not address that in this paper. \\
- Why hasn’t this problem been solved before, or why are existing solutions insufficient? --> seems better suited for Related Work
}

% This format is to be used for submissions that are published in the
% conference proceedings. We wish to give this volume a consistent,
% high-quality appearance. We therefore ask that authors follow some
% simple guidelines. You should format your paper exactly like this
% document. The easiest way to do this is to replace the content with
% your own material.  This document describes how to prepare your
% submissions using \LaTeX.