% !TEX root =  ../geospatial-video.tex

\section{Language}
\todo{to focus: \\
- by-frames operations: as a response to Lisa's use case where we need to scope the output video to the only interesting parts of each videos. \\
- by-frames <--> by-instances: as a response to Yousef's comments to allow the language to express more queries
}

To address the pain points we noticed in our user interviews, we chose to design a new language for describing queries over geospatial video data as a domain-specific language embedded in Python. Instead of building a new language from scratch, which would require designing a new syntax and editor integrations that users would have to learn about, we are able to inherit the large ecosystem of existing tooling around Python. Furthermore, some potential users of our system are already familiar with Python, which makes it further easier to integrate into existing pipelines.

\subsection{Design Guidelines}
There are three key factors constraining the design of our language: incremental query creation, support for using the language through a graphical user interface, and performant query execution. Before we dive into how we design the language around these constraints, let us walk through how we derived the factors based on our user-studies and their consequences on the space of language designs.

\subsubsection{Incremental Queries}
Across the interviews, a key pattern that stuck out was the process by which the participants would identify portions of video that were of interest---either by manual scrubbing through video files or the creation of queries. Rather than define the end-to-end query, the participants would start with a coarse grained, looking for portions of the video that are likely to have the information they are looking for. Across these sections, the participants would then refine their search/query by playing back the video in real-time or querying individual frames in more detail.

Our insight from this observation was that users of video query languages will want to develop the query incrementally, where they can start with a query that identifies sections of the video with general patterns related to their specific goal (such as the presence of multiple cars in the same frame) and iteratively refine it to identify the specific scenario (such as cars moving towards each other) while receiving feedback from the query engine.

\subsubsection{Graphical User Interface}
TODO

\subsubsection{Performant Query Execution}
TODO

\subsection{Language Design}
TODO

\subsection{Prototype Implementation}
TODO?
