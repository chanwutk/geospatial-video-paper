% !TEX root =  ../geospatial-video.tex

\section{Dataset}
%\todo{to focus: \\
%- explain briefly about the dataset, what for, shape of the data.
%- how we transform the dataset to fit our data model.
%- 
%}
%\todo{Amy}

For our prototype, we use the nuScenes dataset \cite{}, a collection of image, lidar, and radar to facilitate autonomous driving research. Data is collected from 15 hours of driving data in Boston and Singapore, and key frames are sampled at a rate of 2Hz for labelling by human annotators, resulting in a rich collection of information suitable for various downstream needs in autonomous driving research. We choose this dataset as its schema is conducive to our aim of.

The entities which the users must interact with are abstracted into \textit{annotation, instance, frame}, and \textit{video}. Specifically, by specifying a string-identifier of a scene, the user can returns \texttt{Video} consisting of a list of \texttt{Frames}. Each \texttt{Frame} consists of several \texttt{instances} accompanied by \texttt{annotations}. This simple interface abstracts away the need to interact with the native nuScenes schema and offers easy access to information which is likely to be used, and furthermore provides a coarse-to-finegrained view of the data, which arose during the user need-finding studies.